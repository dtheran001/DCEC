%&pdflatex

\documentclass[12pt]{article}

\usepackage[spanish, es-tabla]{babel}
\usepackage{enumerate}
\usepackage{lscape}
\usepackage{vmargin}
\usepackage{pdfpages}
\usepackage{fancyhdr}
\usepackage{graphicx}
\usepackage{float}
\usepackage{titlesec}
\usepackage[bottom]{footmisc}
\usepackage[hidelinks]{hyperref}
\usepackage{listings}
\usepackage{color}
\usepackage{colortbl}
\usepackage{xcolor}
\usepackage{amsmath}
\usepackage{array}
\usepackage{svg}
\usepackage{pgfgantt}
\usepackage[T1]{fontenc}
\usepackage[sfdefault]{AlegreyaSans} %% Option 'black' gives heavier bold face
%% The 'sfdefault' option to make the base font sans serif
\renewcommand*\oldstylenums[1]{{\AlegreyaSansOsF #1}}


%*******************************************************************************
\extrarowheight = -0.3ex
\renewcommand{\arraystretch}{2.25}
\setpapersize{A4}
\hypersetup{
    colorlinks=true,
    linkcolor=blue,
    urlcolor=purple,
    citecolor=black, 
    linktocpage=true,
}
\definecolor{gray95}{gray}{.95}
\definecolor{gray75}{gray}{.75}
\definecolor{barblue}{RGB}{153,204,254}
\definecolor{groupblue}{RGB}{51,102,254}
\definecolor{linkred}{RGB}{165,0,33}
\lstset{
    frame=Ltb,
    framerule=0pt,
     aboveskip=0.5cm,
     framextopmargin=3pt,
     framexbottommargin=3pt,
     framexleftmargin=0.4cm,
     framesep=0pt,
     rulesep=.4pt,
     backgroundcolor=\color{gray95},
     rulesepcolor=\color{cyan},
     %
     stringstyle=\ttfamily,
     showstringspaces = false,
     basicstyle=\small\ttfamily,
     commentstyle=\color{cyan},
     keywordstyle=\bfseries\color{purple},
     %
     numbers=left,
     numbersep=15pt,
     numberstyle=\small,
     numberfirstline = false,
     breaklines=true,
}

%minimizar fragmentado de listados
\lstnewenvironment{listing}[1][]
   {\lstset{#1}\pagebreak[0]}{\pagebreak[0]}


\lstdefinestyle{C}
   {
       language=C++,
   }

\lstdefinestyle{python}
    {
        language=Python,
    }

\setcounter{secnumdepth}{4}
%*******************************************************************************
%   Adding a new level of section --> subsubsubsection
%******************************************************************************
\titleclass{\subsubsubsection}{straight}[\subsection]

\newcounter{subsubsubsection}[subsubsection]
\renewcommand\thesubsubsubsection{\thesubsubsection.\arabic{subsubsubsection}}
\renewcommand\theparagraph{\thesubsubsubsection.\arabic{paragraph}} % optional; useful if paragraphs are to be numbered

\titleformat{\subsubsubsection}
  {\normalfont\normalsize\bfseries}{\thesubsubsubsection}{1em}{}
\titlespacing*{\subsubsubsection}
{0pt}{3.25ex plus 1ex minus .2ex}{1.5ex plus .2ex}

\makeatletter
\renewcommand\paragraph{\@startsection{paragraph}{5}{\z@}%
  {3.25ex \@plus1ex \@minus.2ex}%
  {-1em}%
  {\normalfont\normalsize\bfseries}}
\renewcommand\subparagraph{\@startsection{subparagraph}{6}{\parindent}%
  {3.25ex \@plus1ex \@minus .2ex}%
  {-1em}%
  {\normalfont\normalsize\bfseries}}
\def\toclevel@subsubsubsection{4}
\def\toclevel@paragraph{5}
\def\toclevel@paragraph{6}
\def\l@subsubsubsection{\@dottedtocline{4}{7em}{4em}}
\def\l@paragraph{\@dottedtocline{5}{10em}{5em}}
\def\l@subparagraph{\@dottedtocline{6}{14em}{6em}}
\makeatother

\setcounter{secnumdepth}{4}
\setcounter{tocdepth}{4}

%*****************************************************************************

\begin{document}

  \begin{titlepage}
    \centering
   {\bfseries\Large Universidad Carlos III de Madrid \par}
    \vspace{5cm}
    {\scshape\Huge Informe de la tercera práctica de laboratorio\par}
    \vspace{2cm}
    {\itshape\Large Diseño de circuitos electrónicos para comunicaciones}
    \vfill
    {\Large Autores: \par}
    \vspace{1cm}
    {\Large Markel Serrano y Daniel Theran}
    \vfill
    {\Large 19 de diciembre del 2022 \par}
  \end{titlepage}

  \section{Apartado 1}

    \paragraph*{}
    En este apartado se pedía realizar el montaje del siguiente circuito esquemático: 

    \begin{figure}[H]
      \centering
      \includegraphics[width=1\linewidth]{img/1khz.jpeg}
      \caption{Circuito esquemático a montar: Amplificador sintonizado}%
      \label{fig:esque_ampl}
    \end{figure}
    
    \paragraph*{}
    Como de puede observar, se trata de un circuito con una etapa diferencia adaptada (puesto que no tenemos manera de invertir la señal e introducir 
    exactamente las tensiones positivas y negativas en ambos lados del diferencial) que junto con las bobinas y condensador que unen las dos etapas 
    se obtiene un amplificador de tensión sintonizado. La tensión de salida será la resta del lado izquierdo y derecho del condensador $C_1$, la cual 
    podremos medir con las herramientas que nos presta el osciloscopio. 
    
  

\vspace*{1.0cm}

    \begin{table}[H]
      \begin{center}
          \begin{tabular}{| c | c | c | c |}
              \hline
              \rowcolor{black}
              \textcolor{white}{$F_{ref}$ (kHz)} & \textcolor{white}{$F_{VCO}$ (kHz)} & \textcolor{white}{$\Delta \phi$ ($\mu$ s)} & \textcolor{white}{$V_{VCO}$ (V)}\\ \hline
              1.24 & 15.6 & 100 &  2.12\\ \hline
              1.511 & 23.8 & 102 &  2.35\\ \hline
              1.911 & 29.4 & 114 &  2.61\\ \hline
              2.378 & 35.7 & 124 & 2.91\\ \hline
              2.901 & 41.7 & 144 &  3.27\\ \hline
              3.142 & 47.2 & 172 & 3.44\\ 
              \hline 
          \end{tabular}
          \caption{Tabla de resultados reales de la práctica}
          \label{tab:resul_r}
      \end{center}
    \end{table}

  \paragraph*{}
    De la tabla anterior, se puede recalcar que en el polo de funcionamiento del amplificador ($f_0 = 11 kHz$) es cuando el amplificador tiene un mejor desempeño aplicando una ganancia de 
    aproximadamente 30. Si nos vamos alejando poco a poco de dicho polo, vemos como la ganancia cae rápidamente, por lo que podemos intuir que su comportamiento en el dominio de Laplace será 
    un filtro paso banda de segundo orden. 

    \paragraph*{}
    Para ver ésto más en profundidad, en el apartado siguiente se dibujará el diagrama de bode del amplificador y se comparará con su diagrama ideal. 

  \section{Apartado 3}

    \paragraph*{}
    En este apartado se dibujará el diagrama de bode del amplificador sintonizado construido a partir de los valores obtenidos y además se comparará con su función ideal. 

    \paragraph*{}
    El diagrama de bode obtenido del circuito construido es el siguiente: 
   
    \begin{figure}[H]
      \centering
      \includegraphics[width=1\linewidth]{img/2khz.jpeg}
      \caption{Diagrama de bode del circuito construido}%
      \label{fig:bode_real}
    \end{figure}
    
    \paragraph*{}
    Como se puede observar, tenemos un pico de ganancia máxima en la frecuencia del polo de funcionamiento del amplificador, en 11 kHz, alrededor de él la ganancia aplicada 
    recae rápidamente. 

    \paragraph*{}
    Un diagrama de bode de un filtro paso banda de segundo orden ideal, sería el siguiente: 

    \begin{figure}[H]
      \centering
      \includegraphics[width=1\linewidth]{img/1khz.jpeg}
      \caption{Diagrama de bode un filtro paso banda ideal de segundo orden}%
      \label{fig:bode_ideal}
    \end{figure}


  \section{Apartado 4}




\end{document}